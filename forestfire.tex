\PassOptionsToPackage{unicode=true}{hyperref} % options for packages loaded elsewhere
\PassOptionsToPackage{hyphens}{url}
%
\documentclass[
]{article}
\usepackage{lmodern}
\usepackage{amssymb,amsmath}
\usepackage{ifxetex,ifluatex}
\ifnum 0\ifxetex 1\fi\ifluatex 1\fi=0 % if pdftex
  \usepackage[T1]{fontenc}
  \usepackage[utf8]{inputenc}
  \usepackage{textcomp} % provides euro and other symbols
\else % if luatex or xelatex
  \usepackage{unicode-math}
  \defaultfontfeatures{Scale=MatchLowercase}
  \defaultfontfeatures[\rmfamily]{Ligatures=TeX,Scale=1}
\fi
% use upquote if available, for straight quotes in verbatim environments
\IfFileExists{upquote.sty}{\usepackage{upquote}}{}
\IfFileExists{microtype.sty}{% use microtype if available
  \usepackage[]{microtype}
  \UseMicrotypeSet[protrusion]{basicmath} % disable protrusion for tt fonts
}{}
\makeatletter
\@ifundefined{KOMAClassName}{% if non-KOMA class
  \IfFileExists{parskip.sty}{%
    \usepackage{parskip}
  }{% else
    \setlength{\parindent}{0pt}
    \setlength{\parskip}{6pt plus 2pt minus 1pt}}
}{% if KOMA class
  \KOMAoptions{parskip=half}}
\makeatother
\usepackage{xcolor}
\IfFileExists{xurl.sty}{\usepackage{xurl}}{} % add URL line breaks if available
\IfFileExists{bookmark.sty}{\usepackage{bookmark}}{\usepackage{hyperref}}
\hypersetup{
  pdftitle={forestfire},
  pdfauthor={Seyedeh Shaghayegh Rabbanian},
  pdfborder={0 0 0},
  breaklinks=true}
\urlstyle{same}  % don't use monospace font for urls
\usepackage[margin=1in]{geometry}
\usepackage{color}
\usepackage{fancyvrb}
\newcommand{\VerbBar}{|}
\newcommand{\VERB}{\Verb[commandchars=\\\{\}]}
\DefineVerbatimEnvironment{Highlighting}{Verbatim}{commandchars=\\\{\}}
% Add ',fontsize=\small' for more characters per line
\usepackage{framed}
\definecolor{shadecolor}{RGB}{248,248,248}
\newenvironment{Shaded}{\begin{snugshade}}{\end{snugshade}}
\newcommand{\AlertTok}[1]{\textcolor[rgb]{0.94,0.16,0.16}{#1}}
\newcommand{\AnnotationTok}[1]{\textcolor[rgb]{0.56,0.35,0.01}{\textbf{\textit{#1}}}}
\newcommand{\AttributeTok}[1]{\textcolor[rgb]{0.77,0.63,0.00}{#1}}
\newcommand{\BaseNTok}[1]{\textcolor[rgb]{0.00,0.00,0.81}{#1}}
\newcommand{\BuiltInTok}[1]{#1}
\newcommand{\CharTok}[1]{\textcolor[rgb]{0.31,0.60,0.02}{#1}}
\newcommand{\CommentTok}[1]{\textcolor[rgb]{0.56,0.35,0.01}{\textit{#1}}}
\newcommand{\CommentVarTok}[1]{\textcolor[rgb]{0.56,0.35,0.01}{\textbf{\textit{#1}}}}
\newcommand{\ConstantTok}[1]{\textcolor[rgb]{0.00,0.00,0.00}{#1}}
\newcommand{\ControlFlowTok}[1]{\textcolor[rgb]{0.13,0.29,0.53}{\textbf{#1}}}
\newcommand{\DataTypeTok}[1]{\textcolor[rgb]{0.13,0.29,0.53}{#1}}
\newcommand{\DecValTok}[1]{\textcolor[rgb]{0.00,0.00,0.81}{#1}}
\newcommand{\DocumentationTok}[1]{\textcolor[rgb]{0.56,0.35,0.01}{\textbf{\textit{#1}}}}
\newcommand{\ErrorTok}[1]{\textcolor[rgb]{0.64,0.00,0.00}{\textbf{#1}}}
\newcommand{\ExtensionTok}[1]{#1}
\newcommand{\FloatTok}[1]{\textcolor[rgb]{0.00,0.00,0.81}{#1}}
\newcommand{\FunctionTok}[1]{\textcolor[rgb]{0.00,0.00,0.00}{#1}}
\newcommand{\ImportTok}[1]{#1}
\newcommand{\InformationTok}[1]{\textcolor[rgb]{0.56,0.35,0.01}{\textbf{\textit{#1}}}}
\newcommand{\KeywordTok}[1]{\textcolor[rgb]{0.13,0.29,0.53}{\textbf{#1}}}
\newcommand{\NormalTok}[1]{#1}
\newcommand{\OperatorTok}[1]{\textcolor[rgb]{0.81,0.36,0.00}{\textbf{#1}}}
\newcommand{\OtherTok}[1]{\textcolor[rgb]{0.56,0.35,0.01}{#1}}
\newcommand{\PreprocessorTok}[1]{\textcolor[rgb]{0.56,0.35,0.01}{\textit{#1}}}
\newcommand{\RegionMarkerTok}[1]{#1}
\newcommand{\SpecialCharTok}[1]{\textcolor[rgb]{0.00,0.00,0.00}{#1}}
\newcommand{\SpecialStringTok}[1]{\textcolor[rgb]{0.31,0.60,0.02}{#1}}
\newcommand{\StringTok}[1]{\textcolor[rgb]{0.31,0.60,0.02}{#1}}
\newcommand{\VariableTok}[1]{\textcolor[rgb]{0.00,0.00,0.00}{#1}}
\newcommand{\VerbatimStringTok}[1]{\textcolor[rgb]{0.31,0.60,0.02}{#1}}
\newcommand{\WarningTok}[1]{\textcolor[rgb]{0.56,0.35,0.01}{\textbf{\textit{#1}}}}
\usepackage{graphicx,grffile}
\makeatletter
\def\maxwidth{\ifdim\Gin@nat@width>\linewidth\linewidth\else\Gin@nat@width\fi}
\def\maxheight{\ifdim\Gin@nat@height>\textheight\textheight\else\Gin@nat@height\fi}
\makeatother
% Scale images if necessary, so that they will not overflow the page
% margins by default, and it is still possible to overwrite the defaults
% using explicit options in \includegraphics[width, height, ...]{}
\setkeys{Gin}{width=\maxwidth,height=\maxheight,keepaspectratio}
\setlength{\emergencystretch}{3em}  % prevent overfull lines
\providecommand{\tightlist}{%
  \setlength{\itemsep}{0pt}\setlength{\parskip}{0pt}}
\setcounter{secnumdepth}{-2}
% Redefines (sub)paragraphs to behave more like sections
\ifx\paragraph\undefined\else
  \let\oldparagraph\paragraph
  \renewcommand{\paragraph}[1]{\oldparagraph{#1}\mbox{}}
\fi
\ifx\subparagraph\undefined\else
  \let\oldsubparagraph\subparagraph
  \renewcommand{\subparagraph}[1]{\oldsubparagraph{#1}\mbox{}}
\fi

% set default figure placement to htbp
\makeatletter
\def\fps@figure{htbp}
\makeatother


\title{forestfire}
\author{Seyedeh Shaghayegh Rabbanian}
\date{4/15/2021}

\begin{document}
\maketitle

\textbf{Introduction}

Forest fire is a disaster that causes economic and ecological damage and
human life threat. Thus predicting such critical environmental issue is
essential to mitigate this threat. The fire prediction is based on the
meteorological data corresponding to the critical weather elements that
influence the forest fire occurrence, namely temperature, relative
humidity and wind speed.

What is important to us in this study is that how much the classifier
can predict the true cases (fire) correctly.

\textbf{Data description and summary}

The dataset includes 244 instances that regroup a data of two regions of
Algeria,namely the Bejaia region located in the northeast of Algeria and
the Sidi Belabbes region located in the northwest of Algeria. Each
region consist of 122 instances. The data set includes 11 predictors and
one binary response which consists of two classes of ``fire'' and ``not
fire''. The goal is to classify the instances based on certain
predictors correctly. The predictors are as follow: 1. Date :
(DD/MM/YYYY) Day, month (`june' to `september'), year (2012) Weather
data observations 2. Temp : temperature noon (temperature max) in
Celsius degrees: 22 to 42 3. RH : Relative Humidity in \%: 21 to 90 4.
Ws :Wind speed in km/h: 6 to 29 5. Rain: total day in mm: 0 to 16.8 FWI
Components 6. Fine Fuel Moisture Code (FFMC) index from the FWI system:
28.6 to 92.5 7. Duff Moisture Code (DMC) index from the FWI system: 1.1
to 65.9 8. Drought Code (DC) index from the FWI system: 7 to 220.4 9.
Initial Spread Index (ISI) index from the FWI system: 0 to 18.5 10.
Buildup Index (BUI) index from the FWI system: 1.1 to 68 11. Fire
Weather Index (FWI) Index: 0 to 31.1 12. Classes: two classes

First of all we read the dataset. When we check the structure of data,
we can see that there are 4 character variables (date,DC, FWI and
Classes).DC and FWI are indexes and they do not help us in data
analysis. Converting the variables DC, FWI and date to factors lead to
multiple levels which leads to problem in running methods which will be
discussed later. So I decided to remove the variables date, DC and FWI.

\begin{Shaded}
\begin{Highlighting}[]
\NormalTok{forestfire <-}\StringTok{ }\KeywordTok{read.csv}\NormalTok{(}\StringTok{'C:/Users/srabba2/Desktop/forestfire.csv'}\NormalTok{,}\DataTypeTok{header=}\OtherTok{TRUE}\NormalTok{)}
\KeywordTok{str}\NormalTok{(forestfire)}
\end{Highlighting}
\end{Shaded}

\begin{verbatim}
## 'data.frame':    244 obs. of  12 variables:
##  $ date       : chr  "6/1/2012" "6/2/2012" "6/3/2012" "6/4/2012" ...
##  $ Temperature: int  29 29 26 25 27 31 33 30 25 28 ...
##  $ RH         : int  57 61 82 89 77 67 54 73 88 79 ...
##  $ Ws         : int  18 13 22 13 16 14 13 15 13 12 ...
##  $ Rain       : num  0 1.3 13.1 2.5 0 0 0 0 0.2 0 ...
##  $ FFMC       : num  65.7 64.4 47.1 28.6 64.8 82.6 88.2 86.6 52.9 73.2 ...
##  $ DMC        : num  3.4 4.1 2.5 1.3 3 5.8 9.9 12.1 7.9 9.5 ...
##  $ DC         : chr  "7.6" "7.6" "7.1" "6.9" ...
##  $ ISI        : num  1.3 1 0.3 0 1.2 3.1 6.4 5.6 0.4 1.3 ...
##  $ BUI        : num  3.4 3.9 2.7 1.7 3.9 7 10.9 13.5 10.5 12.6 ...
##  $ FWI        : chr  "0.5" "0.4" "0.1" "0" ...
##  $ Classes    : chr  "not fire   " "not fire   " "not fire   " "not fire   " ...
\end{verbatim}

\begin{Shaded}
\begin{Highlighting}[]
\KeywordTok{summary}\NormalTok{(forestfire)}
\end{Highlighting}
\end{Shaded}

\begin{verbatim}
##      date            Temperature          RH              Ws      
##  Length:244         Min.   :22.00   Min.   :21.00   Min.   : 6.0  
##  Class :character   1st Qu.:30.00   1st Qu.:52.00   1st Qu.:14.0  
##  Mode  :character   Median :32.00   Median :63.00   Median :15.0  
##                     Mean   :32.17   Mean   :61.94   Mean   :15.5  
##                     3rd Qu.:35.00   3rd Qu.:73.25   3rd Qu.:17.0  
##                     Max.   :42.00   Max.   :90.00   Max.   :29.0  
##       Rain              FFMC            DMC             DC           
##  Min.   : 0.0000   Min.   :28.60   Min.   : 0.70   Length:244        
##  1st Qu.: 0.0000   1st Qu.:72.08   1st Qu.: 5.80   Class :character  
##  Median : 0.0000   Median :83.50   Median :11.30   Mode  :character  
##  Mean   : 0.7607   Mean   :77.89   Mean   :14.67                     
##  3rd Qu.: 0.5000   3rd Qu.:88.30   3rd Qu.:20.75                     
##  Max.   :16.8000   Max.   :96.00   Max.   :65.90                     
##       ISI              BUI            FWI              Classes         
##  Min.   : 0.000   Min.   : 1.10   Length:244         Length:244        
##  1st Qu.: 1.400   1st Qu.: 6.00   Class :character   Class :character  
##  Median : 3.500   Median :12.25   Mode  :character   Mode  :character  
##  Mean   : 4.774   Mean   :16.66                                        
##  3rd Qu.: 7.300   3rd Qu.:22.52                                        
##  Max.   :19.000   Max.   :68.00
\end{verbatim}

\begin{Shaded}
\begin{Highlighting}[]
\KeywordTok{dim}\NormalTok{(forestfire)}
\end{Highlighting}
\end{Shaded}

\begin{verbatim}
## [1] 244  12
\end{verbatim}

Therefore, I removed those 3 variables and start working on the dataset.

\begin{Shaded}
\begin{Highlighting}[]
\NormalTok{forestfire_withoutfactor <-}\StringTok{ }\KeywordTok{read.csv}\NormalTok{(}\StringTok{'C:/Users/srabba2/Desktop/forestfirewithoutfactor.csv'}\NormalTok{,}\DataTypeTok{header=}\OtherTok{TRUE}\NormalTok{)}
\KeywordTok{dim}\NormalTok{(forestfire_withoutfactor)}
\end{Highlighting}
\end{Shaded}

\begin{verbatim}
## [1] 244   9
\end{verbatim}

\textbf{Procedures for data cleaning and processing}

Next important step in data analysis is data cleaning. First of all, we
need to see if we have any missing values in our dataset.

\begin{Shaded}
\begin{Highlighting}[]
\KeywordTok{sum}\NormalTok{(}\KeywordTok{is.na}\NormalTok{(forestfire_withoutfactor))}
\end{Highlighting}
\end{Shaded}

\begin{verbatim}
## [1] 1
\end{verbatim}

As we can see, there is one missing value in our dataset. We decided to
remove THE instance with missing values.

\begin{Shaded}
\begin{Highlighting}[]
\NormalTok{forestfire_withoutfactor=}\KeywordTok{na.omit}\NormalTok{(forestfire_withoutfactor)}
\KeywordTok{dim}\NormalTok{(forestfire_withoutfactor)}
\end{Highlighting}
\end{Shaded}

\begin{verbatim}
## [1] 243   9
\end{verbatim}

Since one of the observations was removed from the dataset, the
dimension has reduced.

The next step which needs to be done is to change the structure of the
response.The structure of response is converted from chracter to factor
for further analysis.

\begin{Shaded}
\begin{Highlighting}[]
\NormalTok{forestfire_withoutfactor}\OperatorTok{$}\NormalTok{Classes=}\KeywordTok{factor}\NormalTok{(forestfire_withoutfactor}\OperatorTok{$}\NormalTok{Classes)}
\KeywordTok{str}\NormalTok{(forestfire_withoutfactor)}
\end{Highlighting}
\end{Shaded}

\begin{verbatim}
## 'data.frame':    243 obs. of  9 variables:
##  $ Temperature: int  29 29 26 25 27 31 33 30 25 28 ...
##  $ RH         : int  57 61 82 89 77 67 54 73 88 79 ...
##  $ Ws         : int  18 13 22 13 16 14 13 15 13 12 ...
##  $ Rain       : num  0 1.3 13.1 2.5 0 0 0 0 0.2 0 ...
##  $ FFMC       : num  65.7 64.4 47.1 28.6 64.8 82.6 88.2 86.6 52.9 73.2 ...
##  $ DMC        : num  3.4 4.1 2.5 1.3 3 5.8 9.9 12.1 7.9 9.5 ...
##  $ ISI        : num  1.3 1 0.3 0 1.2 3.1 6.4 5.6 0.4 1.3 ...
##  $ BUI        : num  3.4 3.9 2.7 1.7 3.9 7 10.9 13.5 10.5 12.6 ...
##  $ Classes    : Factor w/ 2 levels "fire   ","not fire   ": 2 2 2 2 2 1 1 1 2 2 ...
##  - attr(*, "na.action")= 'omit' Named int 166
##   ..- attr(*, "names")= chr "166"
\end{verbatim}

The next step that needs to be done is to split the dataset into
training and test set. 70\% of the data is randomly chosen for the
training set and 30\% is assigned to the test set. As we previously
disscussed, we decided to remove 3 character variables from the dataset.
So we will work with train\_wof and test\_wof o run the methods.
However, to show the reason of this action, we will run one of the
methods using the dataset including 3 character variables.

\begin{Shaded}
\begin{Highlighting}[]
\KeywordTok{set.seed}\NormalTok{(}\DecValTok{1}\NormalTok{)}
\NormalTok{indx=}\KeywordTok{sample}\NormalTok{ (}\DecValTok{1}\OperatorTok{:}\StringTok{ }\KeywordTok{nrow}\NormalTok{(forestfire), }\DataTypeTok{size=}\FloatTok{0.7}\OperatorTok{*}\KeywordTok{nrow}\NormalTok{(forestfire))}
\NormalTok{train =}\StringTok{ }\NormalTok{forestfire[indx,]}
\NormalTok{test =}\StringTok{ }\NormalTok{forestfire[}\OperatorTok{-}\NormalTok{indx,]}
\end{Highlighting}
\end{Shaded}

\begin{Shaded}
\begin{Highlighting}[]
\KeywordTok{set.seed}\NormalTok{(}\DecValTok{1}\NormalTok{)}
\NormalTok{indx_wof=}\KeywordTok{sample}\NormalTok{ (}\DecValTok{1}\OperatorTok{:}\StringTok{ }\KeywordTok{nrow}\NormalTok{(forestfire_withoutfactor),}
                 \DataTypeTok{size=}\FloatTok{0.7}\OperatorTok{*}\KeywordTok{nrow}\NormalTok{(forestfire_withoutfactor))}
\NormalTok{train_wof =}\StringTok{ }\NormalTok{forestfire_withoutfactor[indx_wof,]}
\NormalTok{test_wof =}\StringTok{ }\NormalTok{forestfire_withoutfactor[}\OperatorTok{-}\NormalTok{indx_wof,]}
\end{Highlighting}
\end{Shaded}

\textbf{Methods}

\textbf{Tree}

We applied different methods including classification tree, bagging,
random forest, MARS and PRIM to our dataset.

\#library(rpart) \#set.seed(1) \#fit1 \textless{}-
rpart(Classes\textasciitilde{}.,train,control=rpart.control(xval=100))
\#fit1 \#print(fit1\(cptable) #which.min(fit1\)cptable{[},``xerror''{]})

When fitting the model with the above way, it took a lot of time to run
the code. Actually, I could not get results because of the high
computation time. For the variables date, DC, and FWI which are factors,
we have too many levels(122, 198, and 126). When splitting a predictor
having q unordered values, there are 2\^{}(q-1)-1 possible partitions
into two groups. Large q leads to severe overfitting. Such variables
should be avoided or collapsed to fewer levels. Since they are date and
index variables, I decided to remove these three variables in my study.

First of all, we applied a classification tree to our dataset. As we can
see, the tree is two shallow and there is no need to prune the tree.
Misclassification rate for the tree method is equal to 0.01369863.

\begin{Shaded}
\begin{Highlighting}[]
\KeywordTok{set.seed}\NormalTok{(}\DecValTok{1}\NormalTok{)}
\KeywordTok{library}\NormalTok{(rpart)}
\end{Highlighting}
\end{Shaded}

\begin{verbatim}
## Warning: package 'rpart' was built under R version 4.0.4
\end{verbatim}

\begin{Shaded}
\begin{Highlighting}[]
\NormalTok{fit1.tree=}\KeywordTok{rpart}\NormalTok{(Classes}\OperatorTok{~}\NormalTok{.,}\DataTypeTok{data=}\NormalTok{train_wof,}\DataTypeTok{control =} \KeywordTok{rpart.control}\NormalTok{(}\DataTypeTok{cp=}\FloatTok{0.001}\NormalTok{,}\DataTypeTok{xval=}\DecValTok{50}\NormalTok{))}

\KeywordTok{printcp}\NormalTok{(fit1.tree)}
\end{Highlighting}
\end{Shaded}

\begin{verbatim}
## 
## Classification tree:
## rpart(formula = Classes ~ ., data = train_wof, control = rpart.control(cp = 0.001, 
##     xval = 50))
## 
## Variables actually used in tree construction:
## [1] FFMC
## 
## Root node error: 77/170 = 0.45294
## 
## n= 170 
## 
##        CP nsplit rel error xerror     xstd
## 1 0.96104      0  1.000000 1.0000 0.084289
## 2 0.00100      1  0.038961 0.1039 0.035858
\end{verbatim}

\begin{Shaded}
\begin{Highlighting}[]
\KeywordTok{par}\NormalTok{(}\DataTypeTok{xpd=}\OtherTok{NA}\NormalTok{)}
\KeywordTok{plot}\NormalTok{(fit1.tree,}\DataTypeTok{uniform =}\NormalTok{ T)}
\KeywordTok{text}\NormalTok{(fit1.tree,}\DataTypeTok{use.n=}\NormalTok{F)}
\end{Highlighting}
\end{Shaded}

\includegraphics{forestfire_files/figure-latex/unnamed-chunk-8-1.pdf}

\begin{Shaded}
\begin{Highlighting}[]
\NormalTok{tree.misrate=}\KeywordTok{rep}\NormalTok{(}\DecValTok{0}\NormalTok{,}\KeywordTok{nrow}\NormalTok{(fit1.tree}\OperatorTok{$}\NormalTok{cptable)}\OperatorTok{-}\DecValTok{1}\NormalTok{)}
\ControlFlowTok{for}\NormalTok{(i }\ControlFlowTok{in} \DecValTok{1}\OperatorTok{:}\NormalTok{(}\KeywordTok{nrow}\NormalTok{(fit1.tree}\OperatorTok{$}\NormalTok{cptable)}\OperatorTok{-}\DecValTok{1}\NormalTok{))\{}
\NormalTok{  prune.fit1=}\KeywordTok{prune}\NormalTok{(fit1.tree,}\DataTypeTok{cp=}\NormalTok{fit1.tree}\OperatorTok{$}\NormalTok{cptable[(i}\OperatorTok{+}\DecValTok{1}\NormalTok{),}\DecValTok{1}\NormalTok{])}
\NormalTok{  pred=}\KeywordTok{predict}\NormalTok{(prune.fit1,}\DataTypeTok{newdata=}\NormalTok{test_wof,}\DataTypeTok{type =} \StringTok{"class"}\NormalTok{)}
\NormalTok{  tab=}\KeywordTok{table}\NormalTok{(pred,test_wof}\OperatorTok{$}\NormalTok{Classes)}
\NormalTok{  tree.misrate=}\DecValTok{1}\OperatorTok{-}\KeywordTok{sum}\NormalTok{(}\KeywordTok{diag}\NormalTok{(tab))}\OperatorTok{/}\KeywordTok{sum}\NormalTok{(tab)}
\NormalTok{\}}

\NormalTok{tree.misrate}
\end{Highlighting}
\end{Shaded}

\begin{verbatim}
## [1] 0.01369863
\end{verbatim}

Then we applied bagging method to our dataset. Bagging misclassification
rate error is equal to 0.01369863.

\textbf{Bagging}

\begin{Shaded}
\begin{Highlighting}[]
\KeywordTok{library}\NormalTok{(nnet)}
\NormalTok{B=}\DecValTok{201}
\NormalTok{n=}\KeywordTok{nrow}\NormalTok{(train_wof)}
\KeywordTok{set.seed}\NormalTok{(}\DecValTok{1}\NormalTok{)}
\NormalTok{bootsamples=}\KeywordTok{rmultinom}\NormalTok{(B,n,}\KeywordTok{rep}\NormalTok{(}\DecValTok{1}\NormalTok{,n)}\OperatorTok{/}\NormalTok{n)}
\NormalTok{trees=}\KeywordTok{vector}\NormalTok{(}\DataTypeTok{mode=}\StringTok{"list"}\NormalTok{,}\DataTypeTok{length=}\NormalTok{B)}
\NormalTok{pred_boot=prob_boot=}\KeywordTok{matrix}\NormalTok{(}\DecValTok{0}\NormalTok{,}\KeywordTok{nrow}\NormalTok{(test_wof),B)}
\NormalTok{fit2_bagging=}\KeywordTok{rpart}\NormalTok{(Classes}\OperatorTok{~}\NormalTok{.,}\DataTypeTok{data=}\NormalTok{train_wof,}\DataTypeTok{control=}\KeywordTok{rpart.control}\NormalTok{(}\DataTypeTok{cp=}\FloatTok{0.001}\NormalTok{,}\DataTypeTok{xval=}\DecValTok{0}\NormalTok{,}\DataTypeTok{maxsurrogate=}\DecValTok{0}\NormalTok{,}\DataTypeTok{maxcompete=}\DecValTok{0}\NormalTok{))}
\ControlFlowTok{for}\NormalTok{(i }\ControlFlowTok{in} \DecValTok{1}\OperatorTok{:}\NormalTok{B)\{}
\NormalTok{  trees[[i]]=}\KeywordTok{update}\NormalTok{(fit2_bagging,}\DataTypeTok{weight=}\NormalTok{bootsamples[,i])}
\NormalTok{  pred_boot[,i]=}\KeywordTok{predict}\NormalTok{(trees[[i]],test_wof,}\DataTypeTok{type=}\StringTok{"class"}\NormalTok{)}
\NormalTok{  prob_boot[,i]=}\KeywordTok{predict}\NormalTok{(trees[[i]],test_wof,}\DataTypeTok{type=}\StringTok{"prob"}\NormalTok{)[,}\DecValTok{2}\NormalTok{]}
\NormalTok{\}}
\NormalTok{bag.vote=}\KeywordTok{apply}\NormalTok{(pred_boot,}\DecValTok{1}\NormalTok{,median)}
\NormalTok{bag.prob=}\KeywordTok{apply}\NormalTok{(prob_boot,}\DecValTok{1}\NormalTok{,mean)}
\NormalTok{bag.prob2=}\KeywordTok{as.numeric}\NormalTok{(bag.prob}\OperatorTok{>=}\FloatTok{0.5}\NormalTok{)}
\NormalTok{tab.bag.vote=}\KeywordTok{table}\NormalTok{(bag.vote,test_wof}\OperatorTok{$}\NormalTok{Classes)}
\NormalTok{tab.bag.prob=}\KeywordTok{table}\NormalTok{(bag.prob2,test_wof}\OperatorTok{$}\NormalTok{Classes)}
\NormalTok{tab.bag.vote; }\DecValTok{1}\OperatorTok{-}\KeywordTok{sum}\NormalTok{(}\KeywordTok{diag}\NormalTok{(tab.bag.vote))}\OperatorTok{/}\KeywordTok{sum}\NormalTok{(tab.bag.vote)}
\end{Highlighting}
\end{Shaded}

\begin{verbatim}
##         
## bag.vote fire    not fire   
##        1      44           1
##        2       0          28
\end{verbatim}

\begin{verbatim}
## [1] 0.01369863
\end{verbatim}

\begin{Shaded}
\begin{Highlighting}[]
\NormalTok{tab.bag.prob; }\DecValTok{1}\OperatorTok{-}\KeywordTok{sum}\NormalTok{(}\KeywordTok{diag}\NormalTok{(tab.bag.prob))}\OperatorTok{/}\KeywordTok{sum}\NormalTok{(tab.bag.prob)}
\end{Highlighting}
\end{Shaded}

\begin{verbatim}
##          
## bag.prob2 fire    not fire   
##         0      44           1
##         1       0          28
\end{verbatim}

\begin{verbatim}
## [1] 0.01369863
\end{verbatim}

From above, we can see the misclassification rate of 201 trees on test
set is 0.01369863 (using averaged probability) is equal to the single
split tree on the original data. We can also calculate the area under
the ROC curve (AUC) and plot the ROC graphs.

\begin{Shaded}
\begin{Highlighting}[]
\KeywordTok{library}\NormalTok{(AUC)}
\end{Highlighting}
\end{Shaded}

\begin{verbatim}
## AUC 0.3.0
\end{verbatim}

\begin{verbatim}
## Type AUCNews() to see the change log and ?AUC to get an overview.
\end{verbatim}

\begin{Shaded}
\begin{Highlighting}[]
\NormalTok{pred.prune.tree=}\KeywordTok{predict}\NormalTok{(}\KeywordTok{prune}\NormalTok{(fit1.tree,}\DataTypeTok{cp=}\FloatTok{0.00100}\NormalTok{),test_wof,}\DataTypeTok{type=}\StringTok{"prob"}\NormalTok{)[,}\DecValTok{2}\NormalTok{]}
\NormalTok{auc.score1=}\KeywordTok{auc}\NormalTok{(}\KeywordTok{roc}\NormalTok{(pred.prune.tree,test_wof}\OperatorTok{$}\NormalTok{Classes))}
\NormalTok{auc.score1}
\end{Highlighting}
\end{Shaded}

\begin{verbatim}
## [1] 0.9827586
\end{verbatim}

\begin{Shaded}
\begin{Highlighting}[]
\NormalTok{auc.score2=}\KeywordTok{auc}\NormalTok{(}\KeywordTok{roc}\NormalTok{(bag.prob,test_wof}\OperatorTok{$}\NormalTok{Classes))}
\NormalTok{auc.score2}
\end{Highlighting}
\end{Shaded}

\begin{verbatim}
## [1] 1
\end{verbatim}

The AUC score for bagging is slightly better than one split
classification tree.

\textbf{Mars}

\begin{Shaded}
\begin{Highlighting}[]
\CommentTok{#set.seed(1)}


\NormalTok{train_wof_num=train_wof}
\NormalTok{test_wof_num=test_wof}


\NormalTok{train_wof_num}\OperatorTok{$}\NormalTok{Classes=}\KeywordTok{as.numeric}\NormalTok{(train_wof}\OperatorTok{$}\NormalTok{Classes)}
\NormalTok{test_wof_num}\OperatorTok{$}\NormalTok{Classes=}\KeywordTok{as.numeric}\NormalTok{(test_wof}\OperatorTok{$}\NormalTok{Classes)}

\NormalTok{train.data.num=}\KeywordTok{as.data.frame}\NormalTok{(train_wof_num)}
\NormalTok{test.data.num=}\KeywordTok{as.data.frame}\NormalTok{(test_wof_num)}

\KeywordTok{library}\NormalTok{(mda)}
\end{Highlighting}
\end{Shaded}

\begin{verbatim}
## Warning: package 'mda' was built under R version 4.0.5
\end{verbatim}

\begin{verbatim}
## Loading required package: class
\end{verbatim}

\begin{verbatim}
## Loaded mda 0.5-2
\end{verbatim}

\begin{Shaded}
\begin{Highlighting}[]
\KeywordTok{set.seed}\NormalTok{(}\DecValTok{1}\NormalTok{)}
\NormalTok{fit1_mars=}\KeywordTok{mars}\NormalTok{(train.data.num[,}\DecValTok{1}\OperatorTok{:}\DecValTok{8}\NormalTok{],train.data.num[,}\DecValTok{9}\NormalTok{])}
\NormalTok{pred1.mars=}\KeywordTok{predict}\NormalTok{(fit1_mars,test.data.num[,}\DecValTok{1}\OperatorTok{:}\DecValTok{8}\NormalTok{])}
\NormalTok{temp1=}\KeywordTok{as.numeric}\NormalTok{(pred1.mars}\OperatorTok{>=}\DecValTok{1}\NormalTok{)}
\NormalTok{res1=}\KeywordTok{table}\NormalTok{(temp1,test.data.num}\OperatorTok{$}\NormalTok{Classes)}
\DecValTok{1}\OperatorTok{-}\KeywordTok{sum}\NormalTok{(}\KeywordTok{diag}\NormalTok{(res1))}\OperatorTok{/}\KeywordTok{sum}\NormalTok{(res1)}
\end{Highlighting}
\end{Shaded}

\begin{verbatim}
## [1] 0.2465753
\end{verbatim}

\begin{Shaded}
\begin{Highlighting}[]
\NormalTok{fit2_mars=}\KeywordTok{mars}\NormalTok{(train.data.num[,}\DecValTok{1}\OperatorTok{:}\DecValTok{8}\NormalTok{],train.data.num[,}\DecValTok{9}\NormalTok{],}\DataTypeTok{degree=}\DecValTok{2}\NormalTok{)}
\NormalTok{pred2.mars=}\KeywordTok{predict}\NormalTok{(fit2_mars,test.data.num[,}\DecValTok{1}\OperatorTok{:}\DecValTok{8}\NormalTok{])}
\NormalTok{temp2=}\KeywordTok{as.numeric}\NormalTok{(pred2.mars}\OperatorTok{>=}\DecValTok{1}\NormalTok{)}
\NormalTok{res2=}\KeywordTok{table}\NormalTok{(temp2,test.data.num}\OperatorTok{$}\NormalTok{Classes)}
\DecValTok{1}\OperatorTok{-}\KeywordTok{sum}\NormalTok{(}\KeywordTok{diag}\NormalTok{(res2))}\OperatorTok{/}\KeywordTok{sum}\NormalTok{(res2)}
\end{Highlighting}
\end{Shaded}

\begin{verbatim}
## [1] 0.1369863
\end{verbatim}

Misclassification rate is high when we apply the method Mars to
degree=1. It will improve when we change the degree to 2.

\textbf{PRIM}

\begin{Shaded}
\begin{Highlighting}[]
\KeywordTok{library}\NormalTok{(prim)}
\end{Highlighting}
\end{Shaded}

\begin{verbatim}
## Warning: package 'prim' was built under R version 4.0.5
\end{verbatim}

\begin{Shaded}
\begin{Highlighting}[]
\KeywordTok{set.seed}\NormalTok{(}\DecValTok{1}\NormalTok{)}
\NormalTok{thr=}\FloatTok{1.43}
\NormalTok{fireforest.prim <-}\StringTok{ }\KeywordTok{prim.box}\NormalTok{(}\DataTypeTok{x=}\NormalTok{train.data.num[,}\DecValTok{1}\OperatorTok{:}\DecValTok{8}\NormalTok{],}\DataTypeTok{y=}\NormalTok{train.data.num[,}\DecValTok{9}\NormalTok{], }\DataTypeTok{peel.alpha=}\FloatTok{0.05}\NormalTok{, }\DataTypeTok{paste.alpha=}\FloatTok{0.01}\NormalTok{,}\DataTypeTok{threshold=}\NormalTok{thr,}\DataTypeTok{threshold.type =} \DecValTok{1}\NormalTok{)}
\NormalTok{pred1.earth=}\KeywordTok{predict}\NormalTok{(fireforest.prim,}\DataTypeTok{newdata=}\NormalTok{test.data.num[,}\DecValTok{1}\OperatorTok{:}\DecValTok{8}\NormalTok{])}
\NormalTok{temp1.prim=}\KeywordTok{as.numeric}\NormalTok{(pred1.earth}\OperatorTok{<=}\DecValTok{1}\NormalTok{)}
\NormalTok{res1.prim=}\KeywordTok{table}\NormalTok{(temp1.prim,test.data.num}\OperatorTok{$}\NormalTok{Classes)}
\DecValTok{1}\OperatorTok{-}\KeywordTok{sum}\NormalTok{(}\KeywordTok{diag}\NormalTok{(res1.prim))}\OperatorTok{/}\KeywordTok{sum}\NormalTok{(res1.prim)}
\end{Highlighting}
\end{Shaded}

\begin{verbatim}
## [1] 0.04109589
\end{verbatim}

I used prim package in order to run prim. The misclassification error
rate for prim model is 0.04109589.

\textbf{Random forest}

\begin{Shaded}
\begin{Highlighting}[]
\KeywordTok{library}\NormalTok{(randomForest)}
\end{Highlighting}
\end{Shaded}

\begin{verbatim}
## Warning: package 'randomForest' was built under R version 4.0.5
\end{verbatim}

\begin{verbatim}
## randomForest 4.6-14
\end{verbatim}

\begin{verbatim}
## Type rfNews() to see new features/changes/bug fixes.
\end{verbatim}

\begin{Shaded}
\begin{Highlighting}[]
\KeywordTok{set.seed}\NormalTok{(}\DecValTok{1}\NormalTok{)}
\NormalTok{fireforest.rf=}\KeywordTok{randomForest}\NormalTok{(Classes}\OperatorTok{~}\NormalTok{.,}\DataTypeTok{data=}\NormalTok{train_wof,}\DataTypeTok{mtry=}\DecValTok{2}\NormalTok{,}\DataTypeTok{importance=}\OtherTok{TRUE}\NormalTok{)}
\KeywordTok{print}\NormalTok{(fireforest.rf)}
\end{Highlighting}
\end{Shaded}

\begin{verbatim}
## 
## Call:
##  randomForest(formula = Classes ~ ., data = train_wof, mtry = 2,      importance = TRUE) 
##                Type of random forest: classification
##                      Number of trees: 500
## No. of variables tried at each split: 2
## 
##         OOB estimate of  error rate: 2.94%
## Confusion matrix:
##             fire    not fire    class.error
## fire             90           3  0.03225806
## not fire          2          75  0.02597403
\end{verbatim}

\begin{Shaded}
\begin{Highlighting}[]
\KeywordTok{library}\NormalTok{(ipred)}
\end{Highlighting}
\end{Shaded}

\begin{verbatim}
## Warning: package 'ipred' was built under R version 4.0.5
\end{verbatim}

\begin{Shaded}
\begin{Highlighting}[]
\KeywordTok{set.seed}\NormalTok{(}\DecValTok{1}\NormalTok{)}
\NormalTok{error.RF=}\KeywordTok{numeric}\NormalTok{(}\DecValTok{20}\NormalTok{)}
\ControlFlowTok{for}\NormalTok{(i }\ControlFlowTok{in} \DecValTok{1}\OperatorTok{:}\DecValTok{20}\NormalTok{)\{}
\NormalTok{  error.RF[i]=}\KeywordTok{errorest}\NormalTok{(Classes}\OperatorTok{~}\NormalTok{.,}\DataTypeTok{data=}\NormalTok{train_wof, }\DataTypeTok{model=}\NormalTok{randomForest,}\DataTypeTok{mtry=}\DecValTok{2}\NormalTok{)}\OperatorTok{$}\NormalTok{error}
\NormalTok{\}}
\KeywordTok{summary}\NormalTok{(error.RF)}
\end{Highlighting}
\end{Shaded}

\begin{verbatim}
##    Min. 1st Qu.  Median    Mean 3rd Qu.    Max. 
## 0.01765 0.02353 0.02941 0.02765 0.02941 0.03529
\end{verbatim}

When I compared the median of 10-fold CV with the OOB estimate of error
rate, I concluded that they are the same(2.94\%).

\begin{Shaded}
\begin{Highlighting}[]
\NormalTok{pred.rf=}\KeywordTok{predict}\NormalTok{(fireforest.rf,test_wof[,}\DecValTok{1}\OperatorTok{:}\DecValTok{8}\NormalTok{])}
\NormalTok{res.rf=}\KeywordTok{table}\NormalTok{(pred.rf,test_wof}\OperatorTok{$}\NormalTok{Classes)}
\DecValTok{1}\OperatorTok{-}\KeywordTok{sum}\NormalTok{(}\KeywordTok{diag}\NormalTok{(res.rf))}\OperatorTok{/}\KeywordTok{sum}\NormalTok{(res.rf)}
\end{Highlighting}
\end{Shaded}

\begin{verbatim}
## [1] 0.02739726
\end{verbatim}

\begin{Shaded}
\begin{Highlighting}[]
\KeywordTok{varImpPlot}\NormalTok{(fireforest.rf)}
\end{Highlighting}
\end{Shaded}

\includegraphics{forestfire_files/figure-latex/unnamed-chunk-15-1.pdf}

\begin{Shaded}
\begin{Highlighting}[]
\KeywordTok{importance}\NormalTok{(fireforest.rf)}
\end{Highlighting}
\end{Shaded}

\begin{verbatim}
##               fire    not fire    MeanDecreaseAccuracy MeanDecreaseGini
## Temperature  3.805533  2.77105039             4.486193        2.7921616
## RH           4.489024  0.74080258             3.825355        1.9503763
## Ws           1.439520 -0.09744555             1.025438        0.5405357
## Rain         8.011715  4.22545093             9.157039        6.5759752
## FFMC        22.251154 19.16201295            25.205002       25.7668718
## DMC          7.396558  4.40870138             8.182468        9.3674552
## ISI         22.679197 19.31694605            25.116849       26.8863878
## BUI          9.677518  5.49113645            11.398995        9.9117031
\end{verbatim}

As we can see, FFMC and ISI are the most two important variables. The
misclassification error rate for random forest method is equal to
0.02739726.

\textbf{SVM for classification with RBF-kernel using cross-validation}

\begin{Shaded}
\begin{Highlighting}[]
\KeywordTok{library}\NormalTok{(e1071)}
\end{Highlighting}
\end{Shaded}

\begin{verbatim}
## Warning: package 'e1071' was built under R version 4.0.5
\end{verbatim}

\begin{Shaded}
\begin{Highlighting}[]
\NormalTok{fireforest.svm=}\KeywordTok{tune.svm}\NormalTok{(Classes}\OperatorTok{~}\NormalTok{.,}\DataTypeTok{data=}\NormalTok{train_wof,}\DataTypeTok{gamma=}\DecValTok{2}\OperatorTok{^}\NormalTok{(}\OperatorTok{-}\DecValTok{2}\OperatorTok{:}\DecValTok{0}\NormalTok{),}\DataTypeTok{cost =} \DecValTok{2}\OperatorTok{^}\NormalTok{(}\DecValTok{0}\OperatorTok{:}\DecValTok{2}\NormalTok{),}\DataTypeTok{sampling=}\StringTok{"cross"}\NormalTok{)}
\KeywordTok{summary}\NormalTok{(fireforest.svm)}
\end{Highlighting}
\end{Shaded}

\begin{verbatim}
## 
## Parameter tuning of 'svm':
## 
## - sampling method: 10-fold cross validation 
## 
## - best parameters:
##  gamma cost
##   0.25    2
## 
## - best performance: 0.07058824 
## 
## - Detailed performance results:
##   gamma cost      error dispersion
## 1  0.25    1 0.07647059 0.05580490
## 2  0.50    1 0.08235294 0.04960436
## 3  1.00    1 0.11764706 0.06200544
## 4  0.25    2 0.07058824 0.05405509
## 5  0.50    2 0.09411765 0.05682893
## 6  1.00    2 0.11176471 0.07568729
## 7  0.25    4 0.08235294 0.06323339
## 8  0.50    4 0.09411765 0.06323339
## 9  1.00    4 0.11764706 0.07336583
\end{verbatim}

\begin{Shaded}
\begin{Highlighting}[]
\KeywordTok{plot}\NormalTok{(fireforest.svm)}
\end{Highlighting}
\end{Shaded}

\includegraphics{forestfire_files/figure-latex/unnamed-chunk-16-1.pdf}

\begin{Shaded}
\begin{Highlighting}[]
\NormalTok{fit.svm=}\KeywordTok{svm}\NormalTok{(train_wof[,}\DecValTok{1}\OperatorTok{:}\DecValTok{8}\NormalTok{],train_wof[,}\DecValTok{9}\NormalTok{])}
\NormalTok{pred.svm=}\KeywordTok{predict}\NormalTok{(fit.svm,test_wof[,}\DecValTok{1}\OperatorTok{:}\DecValTok{8}\NormalTok{])}
\NormalTok{tab.svm=}\KeywordTok{table}\NormalTok{(pred.svm,test_wof}\OperatorTok{$}\NormalTok{Classes)}
\NormalTok{tab.svm}
\end{Highlighting}
\end{Shaded}

\begin{verbatim}
##              
## pred.svm      fire    not fire   
##   fire             40           0
##   not fire          4          29
\end{verbatim}

\begin{Shaded}
\begin{Highlighting}[]
\DecValTok{1}\OperatorTok{-}\KeywordTok{sum}\NormalTok{(}\KeywordTok{diag}\NormalTok{(tab.svm))}\OperatorTok{/}\KeywordTok{sum}\NormalTok{(tab.svm)}
\end{Highlighting}
\end{Shaded}

\begin{verbatim}
## [1] 0.05479452
\end{verbatim}

\begin{Shaded}
\begin{Highlighting}[]
\NormalTok{fit.svm}\OperatorTok{$}\NormalTok{nSV}
\end{Highlighting}
\end{Shaded}

\begin{verbatim}
## [1] 31 32
\end{verbatim}

The misclassification error rate for SVM method is equal to 0.05479452.
Number of support vectors are 31 and 32.

\textbf{Data analysis}

If we compare the misclassification error rates of different methods, we
can see that CART, Bagging, RF, PRIM and SVM have similar performance.
CART and Bagging has exactly same performance. Between all these methods
Mars has the worst performance.

\begin{Shaded}
\begin{Highlighting}[]
\KeywordTok{library}\NormalTok{(knitr)}

\NormalTok{df <-}\StringTok{ }\KeywordTok{data.frame}\NormalTok{(}\DataTypeTok{Method =} \KeywordTok{c}\NormalTok{(}\StringTok{"CART"}\NormalTok{,}\StringTok{"Bagging"}\NormalTok{,}\StringTok{"Mars(default)"}\NormalTok{,}\StringTok{"Mars(deg=2)"}\NormalTok{,}\StringTok{"RF"}\NormalTok{,}\StringTok{"SVM"}\NormalTok{,}\StringTok{"PRIM"}\NormalTok{),}
                 \DataTypeTok{value =} \KeywordTok{c}\NormalTok{(}\FloatTok{0.01369863}\NormalTok{, }\FloatTok{0.01369863}\NormalTok{, }\FloatTok{0.2465753}\NormalTok{, }\FloatTok{0.1369863}\NormalTok{, }\FloatTok{0.02739726}\NormalTok{,}
                           \FloatTok{0.05479452}\NormalTok{,}\FloatTok{0.04109589}\NormalTok{))}

\KeywordTok{print}\NormalTok{(}\KeywordTok{kable}\NormalTok{(df))}
\end{Highlighting}
\end{Shaded}

\begin{verbatim}
## 
## 
## |Method        |     value|
## |:-------------|---------:|
## |CART          | 0.0136986|
## |Bagging       | 0.0136986|
## |Mars(default) | 0.2465753|
## |Mars(deg=2)   | 0.1369863|
## |RF            | 0.0273973|
## |SVM           | 0.0547945|
## |PRIM          | 0.0410959|
\end{verbatim}

\textbf{Results}

We applied differnt methods to our dataset. It seems that our models
perform well in classifying the response variable ``fire'' and ``not
fire''. In CART method, we ran a single split tree. The tree does not
need pruning because the tree is too shallow but successful at
classifying because the misclassification error rate is not too much.
The results gathered from bagging method was same as CART method.
Therefore, there is not much improvement in Bagging method in comparison
to CART. Mars works better when we change the degree to 2 in comparison
to Mars using its default degree value. Random forest, SVM and PRIM
misclassification error rate is low in comparison to MARS. All in all,
CART and Bagging have best performance between all these methods.

\textbf{Discussion and conclusion}

In this study, we presented a comprehensive data analysis using
different methods icluding CART, PRIM, MARS, Bagging, Random forest and
SVM. All the mothods have pretty good performance. CART and BAGGING had
best performance. Mars had worst performance between all these methods.

\end{document}
